\documentclass[]{article}
\usepackage{preamble}
% \usepackage{fullpage}
\usepackage{dsfont}


\title{ \small
University of Oslo\\
FYS4411\\
Computational physics II: Quantum mechanical systems\\
\huge Project 1 }
\author{\textsc{Bendik Samseth}}
\date{\today}
\begin{document}
\maketitle


\section{Introduction}
The aim of this project is to use the Variational Monte Carlo
(VMC) method and evaluate the ground state energy of a trapped, hard
sphere Bose gas for different numbers of particles with a specific
trial wave function.

This trial wave function is used to study the sensitivity of
condensate and non-condensate properties to the hard sphere radius
and the number of particles.  The trap we will use is a spherical (S)
or an elliptical (E) harmonic trap in one, two and finally three
dimensions, with the latter given by

\begin{equation}
    V_{ext}(\mathbf{r}) = 
    \Bigg\{
        \begin{array}{ll}
            \frac{1}{2}m\omega_{ho}^2r^2 & (S)\\
            \strut
            \frac{1}{2}m[\omega_{ho}^2(x^2+y^2) + \omega_z^2z^2] & (E)
            \label{trap_eqn}
        \end{array}
\end{equation}
 where (S) stands for symmetric and

\begin{equation}
    H = \sum_i^N \left(\frac{-\hbar^2}{2m}{\bigtriangledown }_{i}^2 +V_{ext}({\mathbf{r}}_i)\right)  +
     \sum_{i<j}^{N} V_{int}({\mathbf{r}}_i,{\mathbf{r}}_j),
\end{equation}
as the two-body Hamiltonian of the system.  Here $\omega_{ho}^2$
defines the trap potential strength.  In the case of the elliptical
trap, $V_{ext}(x,y,z)$, $\omega_{ho}=\omega_{\perp}$ is the trap
frequency in the perpendicular or $xy$ plane and $\omega_z$ the
frequency in the $z$ direction.  The mean square vibrational
amplitude of a single boson at $T=0K$ in the trap (\ref{trap_eqn}) is
$\langle x^2\rangle=(\hbar/2m\omega_{ho})$ so that $a_{ho} \equiv
(\hbar/m\omega_{ho})^{\frac{1}{2}}$ defines the characteristic length
of the trap.  The ratio of the frequencies is denoted
$\lambda=\omega_z/\omega_{\perp}$ leading to a ratio of the trap
lengths $(a_{\perp}/a_z)=(\omega_z/\omega_{\perp})^{\frac{1}{2}} =
\sqrt{\lambda}$.

We will represent the inter-boson interaction by a pairwise,
repulsive potential

\begin{equation}
    V_{int}(|\mathbf{r}_i-\mathbf{r}_j|) =  \Bigg\{
        \begin{array}{ll}
            \infty & {|\mathbf{r}_i-\mathbf{r}_j|} \leq {a}\\
            0 & {|\mathbf{r}_i-_r\mathbf{r}_j|} > {a}
        \end{array}
\end{equation}
where $a$ is the so-called hard-core diameter of the bosons.
Clearly, $V_{int}(|\mathbf{r}_i-\mathbf{r}_j|)$ is zero if the bosons are
separated by a distance $|\mathbf{r}_i-\mathbf{r}_j|$ greater than $a$ but
infinite if they attempt to come within a distance $|\mathbf{r}_i-\mathbf{r}_j| \leq a$.

Our trial wave function for the ground state with $N$ atoms is given by

\begin{equation}
    \Psi_T(\mathbf{r})=\Psi_T(\mathbf{r}_1, \mathbf{r}_2, \dots \mathbf{r}_N,\alpha,\beta)=\prod_i g(\alpha,\beta,\mathbf{r}_i)\prod_{i<j}f(a,|\mathbf{r}_i-\mathbf{r}_j|),
    \label{eq:trialwf}
\end{equation}
where $\alpha$ and $\beta$ are variational parameters. The
single-particle wave function is proportional to the harmonic
oscillator function for the ground state, i.e.,

\begin{equation}
    g(\alpha,\beta,\mathbf{r}_i)= \exp{[-\alpha(x_i^2+y_i^2+\beta z_i^2)]}.
\end{equation}
For spherical traps we have $\beta = 1$ and for non-interacting
bosons ($a=0$) we have $\alpha = 1/2a_{ho}^2$.  The correlation wave
function is

\begin{equation}
    f(a,|\mathbf{r}_i-\mathbf{r}_j|)=\Bigg\{
        \begin{array}{ll}
            0 & {|\mathbf{r}_i-\mathbf{r}_j|} \leq {a}\\
            (1-\frac{a}{|\mathbf{r}_i-\mathbf{r}_j|}) & {|\mathbf{r}_i-\mathbf{r}_j|} > {a}.
        \end{array}
\end{equation}

We wish to evaluate the expectation value of the Hamiltonian, which we can approximate using the
trial wavefunction as
\begin{align}
    E[H] = \expval{H} = \frac{\int \dd{\vec R} \Psi^*_T H \Psi_T}{\int
    \Psi_T^*\Psi_T}.
\end{align}
The probability of $\vec R$, under the trial wavefunction, is
\begin{align}
    P(\vec R, \alpha) &= \frac{\abs{\Psi_T}^2}{\int \dd{\vec R}\abs{\Psi_T}^2}.
\end{align}
We finally define a new quantity, called the local energy:
\begin{align}
    E_L(\vec R, \vec \alpha) &= \frac{1}{\Psi_T}H\Psi_T\label{eq:E_L}
\end{align}
We can now rewrite $\expval{H}$ as follows:
\begin{align}
    \begin{split}
        \expval{H} &= \int \dd{\vec R} P(\vec R,\vec\alpha) E_L(\vec R,\vec\alpha)\\
    &\approx
    \frac{1}{N}\sum_{i=1}^N E_L(\vec R_i,\vec\alpha),
    \end{split}
\end{align}
where $R_i$ are randomly drawn positions from the PDF $P(\vec R, \vec\alpha)$.
We have therefore that estimating the average value of $E_L$ yields an
approximated value for $\expval{H}$, which must be an upper bound on the
ground state energy, $E_0$. By the variational principle, if we minimize
$\expval{H}$ under the variational parameters, we find an estimate for the true
ground state energy of the system.


\section{Finding an Analytical Expression for $E_L$}
The hard part of the expression for $E_L$ is
\begin{align}
    \frac{1}{\Psi_L}\sum_{k}^N\laplacian_k{\Psi_L}.
\end{align}
To get going, we rewrite the wavefunction as
\begin{align}
    \Psi_L(\vec R) &= \prod_i \phi(\vec r_i)\exp(\sum_{i<j}u(r_{ij})),
\end{align}
where $r_{ij} = \norm{\vec r_{ij}} = \norm{\vec r_i - \vec r_j}$, $u(r_{ij}) =
\ln f(r_{ij})$, and $\phi(\vec r_i)=g(\alpha,\beta,\vec r_i)$.
In order to evaluate this, we first evaluate some expressions we will need.

\begin{align*}
    \grad_k{\Psi_T(\vec r)}  
    &= \grad_k{\prod_i \phi(\vec r_i)\exp(\sum_{i<j}u(r_{ij}))}\\
    &= \prod_{i\neq k} \phi(\vec r_i)\exp(\sum_{i<j}u(r_{ij}))\grad_k{\phi(\vec
    r_k)} + \prod_{i} \phi(\vec r_i)\grad_k{\exp(\sum_{i<j}u(r_{ij}))}\\
    &= \Psi_T\qty[\frac{\grad_k \phi(\vec r_k)}{\phi(\vec r_k)} + \sum_{j\neq k}
    \grad u(r_{kj})].\\
    \grad_k u(r_{kj}) &= u'(r_{kj})\grad_k \sqrt{\norm{\vec r_k - \vec r_j}^2}
    \\
    &= u'(r_{kj})\frac{1}{2r_{kj}}\grad_k\qty(\norm{\vec r_k}^2-2\vec r_k\vdot\vec
    r_j+\norm{r_j}^2)\\
    &= u'(r_{kj})\frac{\vec r_{kj}}{r_{kj}}.\\
    \grad_k\vdot\frac{\vec r_{kj}}{r_{kj}} &= \frac{\grad_k\vdot\qty(\vec r_k - \vec
    r_j) - \qty(\vec r_k-\vec r_j)\vdot\grad_k r_{kj}}{r_{kj}^2}\\
    &= \frac{2}{r_{kj}}.\\
    \laplacian_k u(r_{kj}) &= \grad_k\vdot u'(r_{kj})\frac{\vec
    r_{kj}}{r_{kj}} \\
    &= u'(r_{kj})\frac{2}{r_{kj}} + \frac{\vec r_{kj}}{r_{kj}}\vdot \grad_k
    u'(r_{kj})\\
    &= u''(r_{kj}) + \frac{2}{r_{kj}}u'(r_{kj}).
\end{align*}
\begin{align*}
    u'(r_{ij}) &\equiv \pdv{}{r_{ij}}\ln(1-\frac{a}{r_{ij}})\\
    &= \frac{a}{r_{ij}(r_{ij}-a)}.\\
    u''(r_{ij}) &= \pdv[2]{}{r_{ij}} \ln(1-\frac{a}{r_{ij}})\\
    &= \frac{a(a-2r_{ij})}{r_{ij}^2(r_{ij}-a)^2}.\\
    \frac{\grad_k \phi(\vec r_k)}{\phi(\vec r_k)} &= 
    - 2\alpha\mqty(x\\y\\\beta z).\\
    \frac{\laplacian_k \phi(\vec r_k)}{\phi(\vec r_k)} &=
    2\alpha\qty[2\alpha x^2 + 2\alpha y^2 + 2\alpha\beta^2z^2 - d(\beta)].
\end{align*}
where, $d(\beta) = 1, 2$, or $2 + \beta$ for one, two and three dimensions,
respectively.
Then, with all this in place, we get:
\begin{align*}
    \frac{1}{\Psi_L}\laplacian_k{\Psi_L} &= \frac{1}{\Psi_L}\grad_k\vdot\qty{
    \Psi_T\qty[\frac{\grad_k \phi(\vec r_k)}{\phi(\vec r_k)} + \sum_{j\neq k}
    \grad u(r_{kj})]
    }\\
    &= \frac{\laplacian_k \phi(\vec r_k)}{\phi(\vec r_{k})} +
    \frac{\grad_k\qty(\phi(\vec r_k)) \vdot \qty(\sum_{j\neq k} \grad_k
    u(r_{kj}))}{\phi(\vec r_k)}
    + \sum_{j\neq k}\laplacian_k u(r_{kj})\\
    &+ \qty[
        \qty(\frac{\grad_k \phi(\vec r_{k})}{\phi(\vec r_k)} 
        + \sum_{j\neq k}\grad_k u(r_{kj})    )
        \vdot
        \qty(\sum_{j\neq k} \grad_k u(r_{kj})   )]\\
    &= \frac{\laplacian_k \phi(\vec r_k)}{\phi(\vec r_k)}
    + 2 \frac{\grad_k \phi(\vec r_k)}{\phi(\vec r_k)}\vdot
    \sum_{j\neq k}\qty( \frac{\vec r_{kj}}{r_{kj}}u'(r_{kj}))\\
    &+ \sum_{i,j \neq k} \frac{\vec r_{ki}\vdot\vec r_{kj}}{r_{ki}r_{kj}}
    u'(r_{ki})u'(r_{kj})
    + \sum_{j\neq k}\qty(u''(r_{kj}) + \frac{2}{r_{kj}}u'(r_{kj}))\\
    &= 2\alpha\qty[2\alpha x^2 + 2\alpha y^2 + 2\alpha\beta^2z^2 - \beta - 2]\\
    &- 4\alpha\mqty(x\\y\\\beta z)\vdot
    \qty[ \sum_{j\neq k} \frac{\vec r_{kj}}{r_{kj}}
    \frac{a}{r_{kj}(r_{kj}-a)}]\\
    &+  \sum_{i,j\neq k} \frac{\vec r_{ki}\vdot\vec r_{kj}}{r_{ki}r_{kj}}
    \frac{a}{r_{ki}(r_{ki}-a)}\frac{a}{r_{kj}(r_{kj}-a)}\\
    &+ \sum_{j\neq k} \qty(
    \frac{a(a-2r_{kj})}{r_{kj}^2(r_{kj}-a)^2} +
    \frac{2}{r_{kj}}\frac{a}{r_{kj}(r_{kj}-a)}
    ).\numberthis\label{eq:E_L-part-final}
\end{align*}
We may note that without interactions ($a=0$), this simplifies to only the first
term, as all the other terms are proportional to $a$.

The complete expression for the local energy is then:
\begin{align*}
    E_L &= \frac{1}{\Psi_T}H\Psi_T\\
    &= \sum_{i}V_{ext}(\vec r_i) + \sum_{i<j}V_{int}(\vec r_i,\vec r_j)
    - \frac{\hbar^2}{2m}\sum_k
    \frac{1}{\Psi_T}\laplacian_k\Psi_T\numberthis\label{eq:E_L-final}
\end{align*}
where we substitute in \eqref{eq:E_L-part-final} in the final sum.
\printbibliography

\end{document}
