\documentclass[twocolumn]{article}
\usepackage{preamble}
% \usepackage{fullpage}
\usepackage{dsfont}
\usepackage{algorithm}
\usepackage{algorithmic}
\usepackage{paralist}
\usepackage{csvsimple}

\sisetup{range-phrase=-}


\newcommand{\drawfrom}{\overset{\mathrm{d}}{=}}
\newcommand{\setfrom}{\overset{\mathrm{d}}{\leftarrow}}

\title{ \small
University of Oslo\\
FYS4411\\
Computational physics II: Quantum mechanical systems\\
\huge Project 1 }
\author{\textsc{Bendik Samseth}}
\date{\today}
\begin{document}
\maketitle


\section{Introduction}
The aim of this project is to use the Variational Monte Carlo
(VMC) method and evaluate the ground state energy of a trapped, hard
sphere Bose gas for different numbers of particles with a specific
trial wave function. This trial wave function is used to study the sensitivity of
condensate and non-condensate properties to the hard sphere radius
and the number of particles.  

\section{Theory}
\subsection{Physical System}
\textbf{The trap} we will use is a spherical (S)
or an elliptical (E) harmonic trap in one, two and finally three
dimensions, with the latter given by

\begin{equation}
    V_{ext}(\mathbf{r}) = 
    \Bigg\{
        \begin{array}{ll}
            \frac{1}{2}m\omega_{ho}^2r^2 & (S)\\
            \strut
            \frac{1}{2}m[\omega_{ho}^2(x^2+y^2) + \omega_z^2z^2] & (E)
            \label{trap_eqn}
        \end{array}
\end{equation}
\textbf{The Hamiltonian} of the system will be
\begin{equation}
    H = \sum_i^N \left(\frac{-\hbar^2}{2m}{\bigtriangledown }_{i}^2 +V_{ext}({\mathbf{r}}_i)\right)  +
     \sum_{i<j}^{N} V_{int}({\mathbf{r}}_i,{\mathbf{r}}_j),
\end{equation}
Here $\omega_{ho}^2$ defines the trap potential strength.  In the case of the
elliptical trap, $V_{ext}(x,y,z)$, $\omega_{ho}=\omega_{\perp}$ is the trap
frequency in the perpendicular or $xy$ plane and $\omega_z$ the frequency in
the $z$ direction.  The mean square vibrational amplitude of a single boson at
$T=0K$ in the trap (\ref{trap_eqn}) is $\langle
x^2\rangle=(\hbar/2m\omega_{ho})$ so that $a_{ho} \equiv
(\hbar/m\omega_{ho})^{\frac{1}{2}}$ defines the characteristic length of the
trap.  The ratio of the frequencies is denoted
$\lambda=\omega_z/\omega_{\perp}$ leading to a ratio of the trap lengths
$(a_{\perp}/a_z)=(\omega_z/\omega_{\perp})^{\frac{1}{2}} = \sqrt{\lambda}$.

\vspace{0.3cm}\hrule\vspace{0.2cm}
Note: In the rest of this report, as well as in accompanying source code, we
will use natural units with $\hbar = m = 1$.
\vspace{0.2cm}\hrule\vspace{0.3cm}

We will represent \textbf{the inter-boson interaction} by a pairwise,
repulsive potential:
\begin{equation}
    V_{int}(|\mathbf{r}_i-\mathbf{r}_j|) =  \Bigg\{
        \begin{array}{ll}
            \infty & {|\mathbf{r}_i-\mathbf{r}_j|} \leq {a}\\
            0 & {|\mathbf{r}_i-_r\mathbf{r}_j|} > {a}
        \end{array}
\end{equation}
where $a$ is the so-called hard-core diameter of the bosons.
Clearly, $V_{int}(|\mathbf{r}_i-\mathbf{r}_j|)$ is zero if the bosons are
separated by a distance $|\mathbf{r}_i-\mathbf{r}_j|$ greater than $a$ but
infinite if they attempt to come within a distance $|\mathbf{r}_i-\mathbf{r}_j| \leq a$.

\textbf{The trial wave function} for the ground state with $N$ atoms will be given by
\begin{align}
    \begin{split}
    \Psi_T(\mathbf{r})&=\Psi_T(\mathbf{r}_1, \mathbf{r}_2, \dots
    \mathbf{r}_N,\alpha,\beta)\\
    &=\prod_i g(\alpha,\beta,\mathbf{r}_i)\prod_{i<j}f(a,|\mathbf{r}_i-\mathbf{r}_j|),
    \end{split}
    \label{eq:trialwf}
\end{align}
where $\alpha$ and $\beta$ are variational parameters. We choose the
single-particle wave function to be proportional to the harmonic
oscillator function for the ground state, i.e., we define $g(\alpha,\beta,\mathbf{r}_i)$ as:
\begin{equation}
    g(\alpha,\beta,\mathbf{r}_i)= \exp[-\alpha(x_i^2+y_i^2+\beta z_i^2)].
\end{equation}
For spherical traps we have $\beta = 1$ and for non-interacting
bosons ($a=0$) we have $\alpha = 1/2a_{ho}^2$ resulting in the exact wave
function. The correlation wave function is

\begin{equation}
    f(a,|\mathbf{r}_i-\mathbf{r}_j|)=\Bigg\{
        \begin{array}{ll}
            0 & {|\mathbf{r}_i-\mathbf{r}_j|} \leq {a}\\
            (1-\frac{a}{|\mathbf{r}_i-\mathbf{r}_j|}) & {|\mathbf{r}_i-\mathbf{r}_j|} > {a}.
        \end{array}
\end{equation}

\subsection{The Objective}
Our objective is to evaluate the expectation value of the Hamiltonian. We cannot do this
without the true wave function of the system, something we do not possess.
We can, however, approximate the energy with the trial wave function.
\begin{align}
    E[H] = \expval{H} = \frac{\int \dd{\vec R} \Psi^*_T H \Psi_T}{\int
    \Psi_T^*\Psi_T}.
\end{align}
where $\vec R$ is the matrix containing all the positions of the particles in
the system, $\vec R = [\vec r_1, \vec r_2, \dots, \vec r_N]$.
In order to numerically evaluate this integral we first manipulate it a bit.
The probability density at position $\vec R$, under the trial wave function, is
\begin{align}
    P(\vec R, \vec \alpha) &= \frac{\abs{\Psi_T}^2}{\int \dd{\vec R}\abs{\Psi_T}^2}.
\end{align}
where $\vec \alpha$ is used for shorthand and represents the vector of all the variational parameters.
We finally define a new quantity, called \textbf{the local energy}:
\begin{align}
    E_L(\vec R, \vec \alpha) &= \frac{1}{\Psi_T}H\Psi_T\label{eq:E_L}
\end{align}
Combining these two definitions we can now rewrite $\expval{H}$ as follows:
\begin{align}
    \begin{split}
        \expval{H} &= \int \dd{\vec R} P(\vec R,\vec\alpha) E_L(\vec R,\vec\alpha)\\
        &\approx
        \frac{1}{N}\sum_{i=1}^N E_L(\vec R_i,\vec\alpha),
    \end{split}\label{eq:the-objective}
\end{align}
where $R_i$ are randomly drawn positions from the PDF $P(\vec R, \vec\alpha)$.
We have therefore that estimating the average value of $E_L$ yields an
approximated value for $\expval{H}$. This value is in turn be an upper bound on the
ground state energy, $E_0$. By the variational principle, if we minimize
$\expval{H}$ under the variational parameters, we find an estimate for the true
ground state energy of the system.

\subsubsection{Exact Result for Simple System}

It will be useful to be able to compare our results with exact analytical
results where we have these. In the case of the symmetric harmonic oscillator
trap, ignoring any iteractions between the bosons, we have an exact form for the
ground state energy:
\begin{align}
    E_0 = \sum_{i=1}^N\sum_{d=1}^{D=\{1,2,3\}} \frac{\hbar \omega_{ho}}{2} 
    = \frac{N\times D}{2},\label{eq:exact-ground-state}
\end{align}
for $N$ independent bosons in $D$ dimensions (the two sums goes
over all the degrees of freedom in the system). This follows from the setting
$\alpha=\flatfrac{1}{2}$ (and $\beta=1$) in $\Psi_T$, and using $a=0$ (no
interaction). 

We also have an exact value for the variance of the energy in this case:
\begin{align}
    \begin{split}
        \sigma_E^2 &= \expval{H^2} - \expval{H}^2\\
            &= \expval{H^2}{\Psi}-\expval{H}{\Psi}^2\\
            &= \expval{E^2}{\Psi}-\expval{E}{\Psi}^2\\
            &= E^2\braket{\Psi} - \qty(E\braket{\Psi})^2= 0.
    \end{split}\label{eq:var-zero-when-exact}
\end{align}
This follows when we have the exact wavefunction, which satisfies the time
independent Schrödinger equation, $H\ket{\Psi}=E\ket{\Psi}$.

\subsection{Calculating the Local Energy $E_L$}
As the local energy is the quantity we are interested in computing for a
large set of positions we would do well to consider how best to evaluate this
expression effectively. For this we have two alternative approaches,
\begin{inparaenum}[1)]
    \item numerical differentiation and
    \item finding an analytic, direct expression.
\end{inparaenum}

\subsubsection{Numerical differentiation}
We may set up an algorithm for the numerical approximation of the local energy
as shown in Algorithm~\ref{alg:E_L-numeric}.
\begin{algorithm}[H]
    \caption{Calculate the local energy $E_L$ using numerical differentiation.}
    \label{alg:E_L-numeric}
    \begin{algorithmic}[1]
        \REQUIRE $\vec R = [\vec r_1,\vec r_2,\dots,\vec r_N]$, $D=\text{dimensions}$
        \ENSURE $y = E_L$
        \STATE $y = -2\times N\times D\times  \Psi_T(\vec R)$
        \FOR{$i = 1$ \TO $N$}
            \FOR{$d = 1$ \TO $D$}
                \STATE $\vec R_+\leftarrow \vec R + h \vec e_{i, d}$
                \STATE $\vec R_-\leftarrow \vec R - h \vec e_{i, d}$
                \STATE $y\leftarrow y + \Psi_T(\vec R_+) + \Psi_T(\vec R_-)$
            \ENDFOR
        \ENDFOR
        \STATE $y\leftarrow -\flatfrac{y}{2h^2}$
        \STATE $y\leftarrow \flatfrac{y}{\Psi_T(\vec R)}+\sum_{i=1}^N V_{ext} + \sum_{i<j}^N V_{int} $
    \end{algorithmic}
\end{algorithm}
An evaluation of $E_L$ using this algorithm would be $\mathcal{O}(N^3\times
D)=\mathcal{O}(N^3)$ with interaction, and $\mathcal{O}(N^2)$ without interaction, from
the complexity of $\Psi_T$.



\subsubsection{Finding an Analytical Expression for $E_L$}
Straight forward numerical differentiation is of course an
option, but this is likely to be quite time-expensive to do. We will here try to
speed up the calculation by producing a direct formula.

The hard part of the expression for $E_L$ is
\begin{align}
    \frac{1}{\Psi_L}\sum_{k}^N\laplacian_k{\Psi_L}.
\end{align}
To get going, we rewrite the wave function as
\begin{align}
    \Psi_L(\vec R) &= \prod_i \phi(\vec r_i)\exp(\sum_{i<j}u(r_{ij})),
\end{align}
where $r_{ij} = \norm{\vec r_{ij}} = \norm{\vec r_i - \vec r_j}$, $u(r_{ij}) =
\ln f(r_{ij})$, and $\phi(\vec r_i)=g(\alpha,\beta,\vec r_i)$.

Lets first evaluate the gradient with respect to particle $k$
\begin{align}
    \begin{split}
    \grad_k{\Psi_T(\vec r)}  
    &= \grad_k{\prod_i \phi(\vec r_i)\exp(\sum_{i<j}u(r_{ij}))}\\
    &= \prod_{i\neq k} \phi(\vec r_i)\exp(\sum_{i<j}u(r_{ij}))\grad_k{\phi(\vec
    r_k)}\\
    &+ \prod_{i} \phi(\vec r_i)\grad_k{\exp(\sum_{i<j}u(r_{ij}))}\\
    &= \Psi_T\qty[\frac{\grad_k \phi(\vec r_k)}{\phi(\vec r_k)} + \sum_{j\neq k}
    \grad_k u(r_{kj})].
    \end{split}\label{eq:grad-trail-wf}
\end{align}
The first term is evaluated quite simply:
\begin{align}
    \begin{split}
    \frac{\grad_k\phi(\vec r_k)}{\phi(\vec r_k)} &= \frac{\grad_k}{\phi(\vec r)}
    \exp[-\alpha\qty(x_k^2 + y_k^2 + \beta z_k^2)]\\
    &= \frac{\grad_k}{\phi(\vec r)}\exp[-\alpha\hat{\vec r}_k^2]\\
    &= -2\alpha \hat{\vec r}_k,
    \end{split}\label{eq:grad-phi_k-over-phi_k}
\end{align}
where the notation $\hat{\vec r}_k = (x, y, \beta z)$ is introduced for brevity.
Note that in the 1D and 2D case we simply have $\hat{\vec r}_k = \vec r_k$.


The second term may be evaluated as follows:
\begin{align}
    \begin{split}
    \grad_k u(r_{kj}) &= u'(r_{kj})\grad_k \sqrt{\norm{\vec r_k - \vec r_j}^2} \\
    &= \frac{u'(r_{kj})}{2r_{kj}}\grad_k\qty(\norm{\vec r_k}^2-2\vec r_k\vdot\vec
    r_j+\norm{r_j}^2)\\
    &= u'(r_{kj})\frac{\vec r_{kj}}{r_{kj}}\\
    &= \pdv{}{r_{kj}}\qty[ \ln(1 - \frac{a}{r_{kj}})] \frac{\vec
    r_{kj}}{r_{kj}}\\
    &= \frac{\vec r_{kj}}{r_{kj}}\frac{a}{r_{kj}(r_{kj}-a)}.
    \end{split}\label{eq:grad-u}
\end{align}

Now we can find the Laplacian by taking the divergence of
\eqref{eq:grad-trail-wf}:
\begin{align*}
\frac{1}{\Psi_L}\laplacian_k{\Psi_L} &= \frac{1}{\Psi_L}\grad_k\vdot
    \Psi_T\qty[\frac{\grad_k \phi(\vec r_k)}{\phi(\vec r_k)} + \sum_{j\neq k}
    \grad u(r_{kj})] \\
    &= \frac{\laplacian_k \phi(\vec r_k)}{\phi(\vec r_{k})}+\sum_{j\neq
    k}\laplacian_k u(r_{kj})\\
    &+ \frac{\grad_k\qty(\phi(\vec r_k)) \vdot \qty(\sum_{j\neq k} \grad_k
    u(r_{kj}))}{\phi(\vec r_k)}\\
    &+ \left[
        \qty(\frac{\grad_k \phi(\vec r_{k})}{\phi(\vec r_k)} 
        + \sum_{j\neq k}\grad_k u(r_{kj})    )\right.\\
        &\left.\vdot
        \qty(\sum_{j\neq k} \grad_k u(r_{kj})   )\right] \\
    &= \frac{\laplacian_k \phi(\vec r_k)}{\phi(\vec r_k)}
    + 2 \frac{\grad_k \phi(\vec r_k)}{\phi(\vec r_k)}\vdot
    \sum_{j\neq k}\qty( \frac{\vec r_{kj}}{r_{kj}}u'(r_{kj}))\\
    &+ \sum_{i,j \neq k} \frac{\vec r_{ki}\vdot\vec r_{kj}}{r_{ki}r_{kj}}
    u'(r_{ki})u'(r_{kj})
    + \sum_{j\neq k}\laplacian_k u(r_{kj}).\numberthis\label{eq:almost-done-EL}
\end{align*}
There are two new quantities here which need to be evaluated before we are done:
\begin{align}
    \begin{split}
    \frac{\laplacian_k \phi(\vec r_k)}{\phi(\vec r_k)} &=
    2\alpha\qty[2\alpha \norm{\hat{\vec r}_k}^2 - d(\beta)],\\
    &\text{with } d(\beta) = \begin{cases}
        1 &\qfor \text{1D}\\
        2 &\qfor \text{2D}\\
        2 + \beta &\qfor \text{3D}
    \end{cases},
    \end{split}
\end{align}
and
\begin{align}
    \begin{split}
    \laplacian_k u(r_{kj}) &= \grad_k\vdot u'(r_{kj})\frac{\vec
    r_{kj}}{r_{kj}} \\
    &= u'(r_{kj})\frac{2}{r_{kj}} + \frac{\vec r_{kj}}{r_{kj}}\vdot \grad_k
    u'(r_{kj})\\
    &= u''(r_{kj}) + \frac{2}{r_{kj}}u'(r_{kj}),
    \end{split}
\end{align}
where
\begin{align}
    \begin{split}
    u''(r_{ij}) &= \pdv[2]{}{r_{ij}} \ln(1-\frac{a}{r_{ij}})\\
    &= \frac{a(a-2r_{ij})}{r_{ij}^2(r_{ij}-a)^2}.
    \end{split}
\end{align}

Inserting all of this back into \eqref{eq:almost-done-EL} we get:
\begin{align}
    \begin{split}
    \frac{1}{\Psi_L}\laplacian_k{\Psi_L} &=  
    2\alpha\qty[2\alpha \norm{\hat{\vec r}_k}^2 - d(\beta)]\\
    &- 4\alpha\hat{\vec r}_k \vdot
    \qty[ \sum_{j\neq k} \frac{\vec r_{kj}}{r_{kj}}
    \frac{a}{r_{kj}(r_{kj}-a)}]\\
    &+  \sum_{i,j\neq k} \frac{\vec r_{ki}\vdot\vec r_{kj}}{r_{ki}r_{kj}}
    \frac{a}{r_{ki}(r_{ki}-a)}\frac{a}{r_{kj}(r_{kj}-a)}\\
    &+ \sum_{j\neq k} \qty(
    \frac{a(a-2r_{kj})}{r_{kj}^2(r_{kj}-a)^2} +
    \frac{2}{r_{kj}}\frac{a}{r_{kj}(r_{kj}-a)}).
    \end{split} \label{eq:E_L-part-final}
\end{align}
We may note that without interactions ($a=0$), this simplifies to only the first
term, as all the other terms are proportional to $a$.

The complete expression for the local energy is then:
\begin{align*}
    E_L &= \frac{1}{\Psi_T}H\Psi_T\\
    &= \sum_{i}V_{ext}(\vec r_i) + \sum_{i<j}V_{int}(\vec r_i,\vec r_j)
    - \frac{1}{2}\sum_k
    \frac{1}{\Psi_T}\laplacian_k\Psi_T\numberthis\label{eq:E_L-final}
\end{align*}
where we substitute in \eqref{eq:E_L-part-final} in the final sum. A single
evaluation of the local energy is then $\mathcal{O}(N^3)$ with interaction,
and $\mathcal{O}(N)$ without.

We can see that without interaction we obtain a linear-time expression, compared
to quadratic-time using numerical differentiation. With interaction we have
not been able to improve the complexity in terms of Big-O. This does not,
however, mean that no improvement is obtained. A closer look shows that the
numerical approach uses more evaluations by a constant factor of about three.
This stems from the three wave function evaluations used in the central
difference approximation of the second derivative. The analytic approach is
closer to using a single evaluation, although the exact ratio is hard to define
as the wave function is not directly used here.

In summary, we will expect a significant speedup using the analytic expression
both with and without interaction enabled. 


\subsection{Calculating the Quantum Drift Force}
Anticipating its later use, we will also find an expression for the so called
quantum drift force, which we shall use when we consider importance sampling.
For now, we just give its definition:
\begin{align}
    \vec F_k = \frac{2\grad_k\Psi_T}{\Psi_T}\label{eq:Q-force-def}
\end{align}
This is interpreted as the force acting on particle $k$ due to the trap and/or
presence of other particles. As a physical justification for why $\vec F_k$
takes this form we can see that $\vec F_k$ is
proportional with the gradient of $\Psi_T$, which we can intuitively understand
as a force pushing the particle towards regions of space with higher
probability. 

Luckily this can now be quickly evaluated due to
the results of the previous section,
\begin{align}
    \vec F_k = 2\qty[\sum_{j\neq k}\frac{\vec r_{kj}}{r_{kj}}
    \frac{a}{r_{kj}(r_{kj}-a)} - 2\alpha\hat{\vec
    r}_k].\label{eq:Q-force-explicit}
\end{align}
With interaction this is $\mathcal{O}(N)$, and without ($a=0$) it simplifies to $\mathcal{O}(1)$.

For brevity, we may later use the notation $\vec F(\vec R) = [\vec F_1, \vec
F_2,\dots,\vec F_N]$, denoting the matrix of all the individual forces on each
particle.



\section{Algorithms}
In \eqref{eq:the-objective} we reformulated our objective of estimating the ground state energy of the
system, $\expval{H}$, to minimizing the average local energy, $\expval{E_L(\vec \alpha)}$ w.r. to
the variational parameters $\vec \alpha = (\alpha, \beta)$. Key to this
reformulation was that the local energy had to be sampled with positions $\vec
R$ that where drawn from the PDF $P(\vec R, \vec\alpha)$. As $P$ is far from any
standard PDF with a known inverse CDF, we cannot trivially generate such
samples. In addition, the normalisation term in its denominator is
computationally expensive to compute. These limitations make the
\textit{Metropolis-Hastings algorithm} the obvious choice to solve this problem. This
algorithm provides a way to generate random samples from a PDF where we only
know the probabilities up to a proportionality constant.

\subsection{Simple Metropolis-Hastings}
The algorithm in its
simplest form is described in Algorithm~\ref{alg:metropolis-simple}.

\begin{algorithm}[H]
    \caption{The Metropolis-Hastings algorithm in its simplest form, as it
    pertains to our specific application.}
    \label{alg:metropolis-simple}
    \begin{algorithmic}[1]
        \REQUIRE $M$, generates $M\times N$ samples.
        \ENSURE $\text{samples} \setfrom P(\vec R, \vec\alpha)$
        \STATE $\text{samples} \leftarrow \text{empty list}$
        \STATE $\vec R \leftarrow \text{randomly initialized matrix of positions}$
        \FOR{$M$ iterations}
            \FOR{every particle $i\in[1, N]$}
                \STATE $\Delta \vec{r} \leftarrow \text{random perturbation vector}$
                \STATE $\vec R^*\leftarrow \vec R $
                \STATE $\vec R^*_i \leftarrow \vec R^{*}_i+ \Delta \vec{r}$ 
                \STATE $q \leftarrow \flatfrac{\abs{\Psi_T(\vec R^*)}^2}{\abs{\Psi_T(\vec R)}^2}$
                \STATE $r \setfrom \text{Unif}(0, 1)$
                \IF{ $r\leq q$ }
                    \STATE $\vec R \leftarrow \vec R^*$
                \ENDIF
                \STATE Append $\vec R$ to samples
            \ENDFOR
        \ENDFOR
    \end{algorithmic}
\end{algorithm}

In this algorithm we move around in position space randomly, accepting new
positions biased towards areas of space where $P(\vec R,\vec\alpha)$ is higher.
We choose to move one particle at a time based on computational efficiency, as
recalculating the local energy can be done more easily when we know only one
particle has moved.

With the generated list of positions we may produce and average for the local
energy, and therefore an estimate for an upper bound on the ground state energy. 

\subsubsection{Limitations}
This algorithm has two major drawbacks. Firstly, the samples generated are not
independent. It is quite clear from the algorithm that the probability of
drawing a certain position is highly dependent on what position we were at
previously. The has implications on how we perform our statistical analysis, as
we must attempt to correct for this limitation. More on this in
section~\ref{sec:statistical-analisys}.

Secondly this algorithm will be quite ineffective in that a significant portion
of the suggested moves (new positions, $\vec R^*$ in
Algorithm~\ref{alg:metropolis-simple}) will be rejected. This is because the new
positions are generated at random, which might cause us to wander around in
regions of position space that are of very little significance, and it might
take a while before we (by chance) stumble upon a more high-probability region.
The effect is then a list of samples that may not be an accurate representation
of the PDF we were trying to approximate to begin with. This will be especially
true for smaller sample sizes, were these defects will account for a larger
proportion of the samples.


\subsection{Metropolis-Hastings with Importance Sampling}
The first limitation of the simple algorithm (not i.i.d.) is inherent to this
kind of sampling, and is hard to avoid. We may, however, attempt to remedy the
second limitation by guiding the random walker towards more promising regions of
position space by proposing new positions in a smarter way than purely randomly.

\subsubsection{Physical Motivation of Results}
We will limit our selfs to a superficial derivation, focusing only on giving a
physical motivation for the results we end up using, as it is outside the scope
of this project to derive this rigorously.

\paragraph{Better generation of new positions:}$\,$\\
A reasonable assumption is to say that particles will tend towards regions of
space where $\abs{\Psi_T}^2$ is larger. We may say that this is the result of a
force, namely the quantum drift force given in \eqref{eq:Q-force-def}, as we
know this force pushes particles towards regions where $\Psi_T$ is large. In
addition, as this a quantum system, we expect some degree of random motion as
well. This intuitive view is exactly what is described by
the Langevin equation,
\begin{align}
    \pdv{\vec r_k}{t} = D\vec F_k(\vec r_k) + \vec\eta,\label{eq:Langevin}
\end{align}
which describes how the position of a particle changes with time. Here, $D$ is
a constant scalar referred to as the drift coefficient. We set
$D=\flatfrac{1}{2}$, originating from the same factor in the kinetic energy. The
term $\vec\eta$ is a vector of
uniformly distributed random values, giving the particle some random motion in each
dimension.

Solving the Langevin equation we can obtain new positions at some time
$t+\Delta t$. Using Euler's method we get:
\begin{align}
    \vec r^* = \vec r + \frac{1}{2}\vec F_k(\vec r_k)\Delta t + \vec \xi\sqrt{\Delta
    t}\label{eq:Langevin-solution},
\end{align}
given a time step $\Delta t$, and where $\vec \xi$ is the normally distributed
equivalent of $\vec \eta$. 

\paragraph{Adjusting the acceptance probability:}$\,$\\
In Algorithm~\ref{alg:metropolis-simple} the acceptance probability for a new
position $\vec R^*$ was
\begin{align}
    q(\vec R^*, \vec R) = \frac{\abs{\Psi_T(\vec R^*)}^2}{\abs{\Psi_T(\vec
    R)}^2}\label{eq:q-simple}.
\end{align}
This was based on the transition probability for going to $\vec R^*$ from $\vec
R$ being uniform. When the transition probabilities are different depending on 
where we are and where we want to go, the acceptance probability has to be
modified in order for the algorithm to work as advertised. In general, we have 
the following form for $q(\vec R^*, \vec R)$ in our case:
\begin{align}
    q(\vec R^*, \vec R) &= \frac{T_{j\rightarrow i}}{T_{i\rightarrow j}}\frac{\abs{\Psi_T(\vec R^*)}^2}{\abs{\Psi_T(\vec
    R)}^2},\label{eq:q-full}
\end{align}
where $T_{i\rightarrow j}$ is the transition probability from the original state
$i$ into the new state $j$. We see that a uniform transition probability gives
us back \eqref{eq:q-simple}. We need therefore an expression for the transition
probabilities when we use the improved generation of new positions.


To this end, we consider the \textit{Fokker-Planck} equation, which for one
dimension and one particle can be written as
\begin{align}
    \pdv{\Psi_T}{t} = D\pdv{}{x}\qty(\pdv{}{x} - F)\Psi_T,\label{eq:Fokker-Planck}
\end{align}
which describes the time-evolution of a probability distribution under the
influence of a drift force and random impulses. We let $\Psi_T$ play the role of the
probability distribution. The factor $D$ is as before, and $F$ is here the
one-dimensional, one-particle analog to $\vec F$. In
fact, this equation is the origin of the specific form of $\vec F$ presented in
\eqref{eq:Q-force-def}. 

Equation \eqref{eq:Fokker-Planck} yields a solution given by
the following Green's function (for one particle):
\begin{align}
    G(\vec r_k^*, \vec r_k, \Delta t) &\propto 
    \exp[- \frac{\norm{\vec r_k^* - \vec r_k - D\Delta t \vec F_k(\vec r_k)}^2}{4D\Delta t} ].
\end{align}
This is interpreted as the probability of transitioning to position $\vec r_k^*$
from position $\vec r$ in a time interval $\Delta t$. Equation~\eqref{eq:q-full}
needs only the ratio of these probabilities, so we may simplify this as
\begin{align}
    \frac{G(\vec r_k, \vec r_k^*, \Delta t)}{G(\vec r_k^*, \vec r_k, \Delta t)} = 
    \exp\left.\begin{cases}
        \qty[\vec F_k(\vec r_k) + \vec F_k(\vec r_k^*)]\\
        \qqtext{}\qqtext{}\bullet\\
        \left[D\Delta t \qty(\vec F_k(\vec r_k)-\vec F_k(\vec r_k^*)) \right.\\
        \left. + 2\vec r_k - 2\vec r_k^*\right]
    \end{cases}\right\}\label{eq:Greens-ratio}
\end{align}
We may then finally insert this ratio into~\eqref{eq:q-full}.



\subsubsection{Improved Algorithm}
We are now ready to present the proper Metropolis-Hastings algorithm, with
importance sampling used to increase the number of accepted transitions.

\begin{algorithm}[H]
    \caption{The Metropolis-Hastings algorithm, with importance sampling, as it
    pertains to our specific application.}
    \label{alg:metropolis-importance}
    \begin{algorithmic}[1]
        \REQUIRE $M$, generates $M\times N$ samples.
        \ENSURE $\text{samples} \setfrom P(\vec R, \vec\alpha)$
        \STATE $\text{samples} \leftarrow \text{empty list}$
        \STATE $\vec R \leftarrow \text{randomly initialized matrix of positions}$
        \FOR{$M$ iterations}
            \FOR{every particle $i\in[1, N]$}
                \STATE $\Delta\vec r_i^* \leftarrow \frac{1}{2}\vec
                F_i(\vec r_i)\Delta t + \vec \xi\sqrt{\Delta t}$
                \STATE $\vec R^*\leftarrow \vec R $
                \STATE $\vec R^*_i \leftarrow \vec R^{*}_i+ \Delta \vec r_i^*$ 
                \STATE $q \leftarrow \flatfrac{\abs{\Psi_T(\vec R^*)}^2}{\abs{\Psi_T(\vec R)}^2}$
                \STATE $q \leftarrow q \times \flatfrac{G(\vec r_k, \vec r_k^*,
                    \Delta t)}{G(\vec r_k^*, \vec r_k, \Delta t)}$
                \STATE $r \setfrom \text{Unif}(0, 1)$
                \IF{ $r\leq q$ }
                    \STATE $\vec R \leftarrow \vec R^*$
                \ENDIF
                \STATE Append $\vec R$ to samples
            \ENDFOR
        \ENDFOR
    \end{algorithmic}
\end{algorithm}


\section{Results}

\subsection{Standard Metropolis VMC}

We now apply the standard Metropolis sampling algorithm described in 
Algorithm\ref{alg:metropolis-simple} to a variety of different settings.
Table~\ref{tab:simple-metro} shows selected results. 

We see that all the results have $\expval{E_L}$
equal (within numerical precision) to the exact result. In addition, the variance is also equal to
or extremely close to zero, as we expect to observe when we use the correct form
for the exact wavefunction. This is strong evidence that the codes developed are
correct.

We can see the already motioned downside of using the simple Metropolis
algorithm in the reported acceptance rate. We can see that about
$\SIrange{15}{25}{\percent}$ of the suggested moves are dropped, which means a
significant amount of work has been wasted.

Perhaps most interesting is to look at the time spent on each configuration. If
we compare the durations for $N=100$ and $N=500$ with analytical expressions
turned on, we see the time increasing by a factor of~$\sim 25$, which is the
square of the factor we increased $N$ by. Looking at the expressions involved in
Algorithm~\ref{alg:metropolis-simple}, we see that this is exactly what we would
expect.

Similarly, comparing the run times with numerical differentiation, we observe
that time scales as $\mathcal{O}(N^3)$, resulting in quite substantial running
times. It is quite clear that the use of analytic expressions is far superior.
We can, however, remember to expect the difference to be reduced when we include
interaction, as this reduced the analytical approach to the same complexity as
the numerical one.


\begin{table*}
    \centering
    \caption{Selected results using the standard Metropolis sampling algorithm.
    All runs have been made with $\alpha=\flatfrac{1}{2}$, $\beta=1$, with a
    symmetric trap with strength $\omega_{ho}=1$ and 100 MC cycles.}
    \label{tab:simple-metro}
    \begin{tabular}{|l|c|c|c|c|c|c|}\hline%
        \bfseries Analytic & \bfseries Dimensions & \bfseries particles & 
        $\expval{E_L}$ & $\text{Var}(E_L)$ & Acceptance Rate & Time (ms)
        \csvreader[head to column names]{../results/vmc_configurations.csv}{}
        {\\\hline\csvcoliii & \csvcoli & \csvcolii & \csvcoliv & \csvcolvi &
        \csvcolix & \csvcolx}\\\hline
    \end{tabular}
\end{table*}




\printbibliography

\end{document}
