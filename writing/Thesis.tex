\documentclass[twoside,english]{uiofysmaster}

\usepackage{preamble}
\usepackage{subfiles}
\usepackage{datetime}
\newdateformat{monthyeardate}{%
  \monthname[\THEMONTH], \THEYEAR}


\author{Bendik Samseth}
\title{(Title Tinkering Needed)Bootstrapping Variational Monte Carlo with Machine Learning}
\date{\monthyeardate\today}

\usepackage{xparse}
\DeclareDocumentCommand{\newdualentry}{ O{} O{} m m m m } {
  \newglossaryentry{gls-#3}{name={#5},text={#5\glsadd{#3}},
    description={#6},#1
  }
  \makeglossaries
  \newacronym[see={[Glossary:]{gls-#3}},#2]{#3}{#4}{#5\glsadd{gls-#3}}
}
\let\firstchar\lowercase
\let\oldprintglossaries\printglossaries
\def\printglossaries{\let\firstchar\uppercase\oldprintglossaries}


\newacronym{mc}{MC}{Monte Carlo}
\newacronym{mci}{MCI}{Monte Carlo integration}
\newacronym[description=Importance sampling]{is}{IS}{importance sampling}
\newacronym[description=Variational Monte Carlo]{vmc}{VMC}{variational Monte Carlo}
\newacronym[description=Diffusion Monte Carlo]{dmc}{DMC}{diffusion Monte Carlo}
\newacronym[description=Machine learning]{ml}{ML}{machine learning}
\newacronym[description=Quantum mechanics]{qm}{QM}{quantum mechanics}
\newacronym[description=Magnetic resonance imaging]{mri}{MRI}{magnetic resonance imaging}
\newacronym[description=Light-emitting diode]{led}{LED}{light-emitting-diode}
\newacronym[description=Time-dependent Schrödinger equation]{tdse}{TDSE}{time-dependent Schrödinger equation}
\newacronym[description=Time-independent Schrödinger equation]{tise}{TISE}{time-independent Schrödinger equation}
\newacronym[description=Standard error of the mean]{sem}{SEM}{standard error of the mean}
\newacronym[description=Acceptance rate]{ar}{AR}{acceptance rate}
\newacronym[description=Central processing unit]{cpu}{CPU}{central processing unit}
\newacronym[description=Graphics processing unit]{gpu}{GPU}{graphics processing unit}
\newacronym[description=Message passing interface]{mpi}{MPI}{message passing interface}
\newacronym[description=Automatic differentiation]{ad}{AD}{automatic differentiation}
\newacronym[description=Restricted Boltzmann machine]{rbm}{RBM}{restricted Boltzmann machine}
\newacronym[description=Mean squared error]{mse}{MSE}{mean squared error}
\newacronym[description=Gradient decent]{gd}{GD}{gradient decent}
\newacronym[description=Stochastic gradient decent]{sgd}{SGD}{stochastic gradient decent}
\newacronym[description=Artificial neural network]{ann}{ANN}{artificial neural network}
\newacronym[description=Neural network]{nn}{NN}{neural network}
\newacronym[description=Convolutional neural network]{cnn}{CNN}{convolutional neural network}
\newacronym[description=Feed-forward neural network]{ffnn}{FFNN}{feed-forward neural network}
\newacronym[description=Deep neural network]{dnn}{DNN}{deep neural network}
\newacronym[description=Application programming interface]{api}{API}{application programming interface}
\newacronym[description=Cumulative distribution function]{cdf}{CDF}{cumulative distribution function}
\newacronym[description=Probability density function]{pdf}{PDF}{probability density function}
\newacronym[description=Linear congruential generator]{lcg}{LCG}{linear congruential generator}
\newacronym[description=Quantum dots]{qd}{QD}{quantum dots}
\makeglossaries

\begin{document}

\maketitle

\begin{abstract}
Recent applications of machine learning for quantum mechanics have shown
encouraging results in efforts to overcome the exponential scaling complexity of
the many-body wave function. We continue this exploration by introducing a
neural network as an additional Jastrow factor to the wave function ansatz
within the framework of variational Monte Carlo, with the hope of learning
correlations beyond what traditional methods have been able to.

We begin with a review of the relevant parts of quantum mechanics with particular
emphasis on Monte Carlo methods. We then introduce central elements of machine
learning, focusing on neural networks. Finally we discuss implementation
challenges.

We test our approach first on two interacting electrons in a harmonic oscillator
potential. Where the benchmark estimates $\expval{E_0}=\SI{3.00064(4)}{\au}$, we
outperform this by a lowering of more than an order of magnitude, resulting in
$\expval{E_0}=\SI{3.000021(4)}{\au}$. The energy estimates approach those of
diffusion Monte Carlo, which can produce exact solutions to the Schrödinger equation.

Second we apply the same technique to the strongly correlated system of liquid
$^4$He. Again we obtain significant lowering of the ground state energy, from
the benchmark result of $\expval{E_0}=\SI{6.76(2)}{\kelvin}$ to
$\expval{E_0}=\SI{6.96(2)}{\kelvin}$, although the optimization problem proves far
more challenging.

These improvements come at the cost of greatly increased computing time,
however, we argue that the time complexity can still be made to scale similarly to
$\mathcal{O}(N^2)$ with the number of particles.
\end{abstract}



\begin{acknowledgements}
  I acknowledge my acknowledgements.
\end{acknowledgements}

\tableofcontents

\listoffigures
\begingroup
\let\clearpage\relax
\listoftables
\endgroup

\subfile{Introduction.tex}

\part{Theory}
\label{prt:theory}
\subfile{QuantumTheory.tex}
\subfile{VariationalMonteCarlo.tex}
\subfile{MonteCarlo.tex}
\subfile{MachineLearning.tex}
\subfile{MergingVMCandML.tex}
\part{Implementation}
\label{prt:implementation}
\subfile{Parallelization.tex}
\subfile{AutomaticDifferentiation.tex}
\subfile{Qflow.tex}
\subfile{Verification.tex}
\part{Results}
\label{prt:results}
\subfile{QuantumDots.tex}
\subfile{LiquidHelium.tex}
\part{Conclusion and Outlook}
\label{prt:conclusion}
\subfile{Conclusion.tex}
\subfile{Appendix.tex}
\printglossary[title=List of Accronyms, toctitle=Terms and abbreviations,type=\acronymtype]
% \begingroup
% \let\clearpage\relax
% \printglossary[title=List of Terms]
% \endgroup

\printbibliography[heading=bibintoc, title={References}]


\end{document}
