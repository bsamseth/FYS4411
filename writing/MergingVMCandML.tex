\documentclass[Thesis.tex]{subfiles}
\begin{document}
\chapter{Merging Variational Monte Carlo and Machine Learning}
\label{chp:mergin-vmc-with-ml}


We have encountered one machine learning algorithm already, namely Variational
Monte Carlo. In VMC, we desire a function that emulates the correct wave
function as best as possible. In order to do this we defined a parameterized
model, a metric by which we determine the goodness of the wave function
(smallest possible expected energy) and set the model to adapt to find the best
set of parameters. By the definitions of machine learning given here, VMC is
fully encompassed by the ML umbrella. Still, it seems that up until recently
physicists have not identified VMC as such, and largely stick to hand-crafted
trial wave functions with only a very select few free parameters and primitive
optimization strategies.

In more recent years, more and more research has gone into applying various
techniques from the more general field of ML into VMC. A notable example
is~\textcite{Carleo602}, who demonstrated that a Restricted Boltzmann Machine
(RBM) was capable of representing the wave function for some notable systems to
great effect.


However, in my opinion that much more of the advances seen in ML in
general could be used also for VMC. I believe that part of the reason why there
seems to be hestitatio n




\end{document}
