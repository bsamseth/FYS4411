\documentclass[Thesis.tex]{subfiles}
\begin{document}
\chapter{Verification}
\label{chp:verfication}

This chapter is dedicated to presenting some of the tests that we have done to
verify that the software we have developed is indeed functioning as advertised.

\section{Setup}

We focus our tests on the idealized harmonic oscillator system in $D = 3$ dimensions
with $N$ non-interacting particles, i.e.\ the Hamiltonian given by
\cref{eq:ho-no-interaction-hamiltonian}:

\begin{align}
    \hat H_0 &= \sum_{i=1}^N\qty(-\frac{1}{2}\laplacian_i +
    \frac{1}{2}r_i^2),
\end{align}
where we have $\hbar = m = \omega = 1$, and
$r_i^2 = x_i^2 + y_i^2+z_i^2$. This has the ground state given by
\cref{eq:Phi-non-inter}, generalized to $N$ particles in three dimension and
omitting normalization constants:

\begin{align}
        \Phi(\mat X) &= \exp[-\frac{1}{2}\sum_{i=1}^N r_i^2],
\end{align}
where as before we have defined $\mat X$ as

\begin{align}
  \mat X &\defeq \mqty(\vx_1\\\vx_2\\\vdots\\\vx_N) \defeq \mqty(x_1&y_1&z_1\\x_2&y_2&z_2\\\vdots&\vdots&\vdots\\x_N&y_N&z_N)
\end{align}

For the trail wave function we shall use two different ones. First, the simple
Gaussian form of the ground state it self:

\begin{align}
  \psi_G(\mat X) = \exp(-\alpha\sum_{i = 1}^N r_i^2),
\end{align}
with $\alpha$ the only variational parameter. Learning the ideal parameters
should be trivial in this case, and we should expect perfect results.

Second, we use an ansatz resulting from a Gaussian-Binary Restricted Boltzmann
Machine (RBM), given in \cref{eq:rbm-def}:


\begin{align}
  \psi_{RBM}(\vX) &=
        \exp[-\sum_i^{M} \frac{\qty(X_i-a_i)^2}{2\sigma^2}]
        \prod_j^H \qty(1 + \exp[b_j+\sum_i^M \frac{X_iW_{ij}}{\sigma^2}]),
\end{align}
where $M = P\cdot D$ is the number of degrees of freedom and $H$ is
the number of hidden nodes (set to 4 through this section). Note also that $X_i$
in the above refers to the $i$'th degree of freedom, counting through $\mat X$
in row major order. The variational parameters are $\vb a, \vb b$ and $\mat W$,
and we hold $\sigma^2=1$ constant in this case.

We use this wave function simply to make learning the true ground state slightly
more challenging than proposing a simple Gaussian straight away. Note that
setting $\vb{a}, \vb b$ and $\mat W$ all to zero yields the correct ground state in this particular case.


\section{Optimization}

The simplest possible test is to initialize $\psi_G$ with a non-optimal
parameter, e.g.\ $\alpha=0.8$.






\end{document}
