\documentclass[Thesis.tex]{subfiles}
\begin{document}
\chapter{Liquid $^4$He}
\label{chp:liquid-helium}

We turn now to the much more challenging system of liquid Helium, presented
in~\cref{sec:liquid-helium-theory}. As before, we will first present the
benchmark result followed by the neural networks.


\section{Benchmark}

We will use one of the simpler benchmark wave functions for this system, known
as the McMillan form wave function~\cite{McMillan-1965}:

\begin{align}
  \label{eq:McMillan-wave-function-def}
  \psi_{M} &= \exp(-\frac{1}{2}\sum_{i<j} \qty(\frac{\beta}{r_{ij}})^5).
\end{align}
An important observation now is the lack of any single particle wave
function factor. In the case of the Quantum Dot we had a Gaussian localized at
the origin as a result of the potential well. This system, however, is infinite
and periodic without any such influence driving it towards particular points in
space. Furthermore, because of the lack of an external field the single particle
solutions are just free particles, and does not help us understand the many-body
system.

An important aspect of all the results that we will present is that they are
highly dependent on the number of particles used in the simulation box, as well
as the size of the box it self. We will hold the number density of particles
constant, $\rho$, and set the side lengths of the simulation box, $L$, depending on
the number of particles $N$:

\begin{align}
  L = \sqrt[3]{\frac{N}{\rho}}.
\end{align}

\noindent As the assumption of periodicity is a simplifying approximation, we
introduce some erroneous effects because of it. These generally disappear as we
increase the number of particles (and hence the size of box), but the
computation time needed to run the simulations increase significantly with
increasing numbers. The purpose of the following analysis is to test the
\emph{relative} accuracy of different wave functions. With that in mind we have
used a small number of particles in the main results, where the absolute error
introduced is significant. The number should hopefully still be large enough to
introduce all the relevant effects and produce valid test results.

\begin{itemize}
\item Show plot of training for 32 particles
\item In table of energies, include same wave function on larger numbers (64,
  128, 256)
\end{itemize}


\end{document}
