\documentclass[Thesis.tex]{subfiles}
\begin{document}
\chapter{Liquid $^4$He}
\label{chp:liquid-helium}

We turn now to the much more challenging system of liquid Helium, presented
in~\cref{sec:liquid-helium-theory}. As before, we will first present the
benchmark result followed by the neural networks.


\section{Benchmark}

We will use one of the simpler benchmark wave functions for this system, known
as the McMillan form wave function~\cite{McMillan-1965}:

\begin{align}
  \label{eq:McMillan-wave-function-def}
  \psi_{M} &= \exp(-\frac{1}{2}\sum_{i<j} \qty(\frac{\beta}{r_{ij}})^5).
\end{align}
An important observation now is the lack of any single particle wave
function factor. In the case of the Quantum Dot we had a Gaussian localized at
the origin as a result of the potential well. This system, however, is infinite
and periodic and as such it would not make sense to include a localized factor.
Furthermore, because of the lack of an external field the single particle
solutions are just free particles, and does not help us understand the many-body
system.

An important aspect of all the results that we will present is that they are
highly dependent on the number of particles used in the simulation box. The
results generally converge with increasing number of particles, but the
computation time needed to run the simulations increase significantly with
increasing numbers.

\begin{itemize}
\item Show plot of training for 32 particles
\item In table of energies, include same wave function on larger numbers (64,
  128, 256)
\end{itemize}


\end{document}
