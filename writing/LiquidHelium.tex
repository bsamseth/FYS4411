\documentclass[Thesis.tex]{subfiles}
\begin{document}
\chapter{Liquid $^4$He}
\label{chp:liquid-helium}

We turn now to the much more challenging system of liquid Helium, presented
in~\cref{sec:liquid-helium-theory}. As before, we will first present the
benchmark result followed by the neural networks.


\section{Benchmark}

We will use one of the simpler benchmark wave functions for this system, known
as the McMillan form wave function~\cite{McMillan-1965}:

\begin{align}
  \label{eq:McMillan-wave-function-def}
  \psi_{M} &= \exp(-\frac{1}{2}\sum_{i<j} \qty(\frac{\beta}{r_{ij}})^5).
\end{align}
An important observation now is the lack of any single particle wave
function factor. In the case of the Quantum Dot we had a Gaussian localized at
the origin as a result of the potential well. This system, however, is infinite
and periodic without any such influence driving it towards particular points in
space. Furthermore, because of the lack of an external field the single particle
solutions are just free particles, and does not help us understand the many-body
system.

\subsection{Finite Size Dependency}

An important aspect of all the results that we will present is that they are
highly dependent on the number of particles used in the simulation box, as well
as the size of the box it self. We will hold the number density of particles
constant, $\rho$, and set the side lengths of the simulation box, $L$, depending on
the number of particles $N$:

\begin{align}
  L = \sqrt[3]{\frac{N}{\rho}}.
\end{align}

\noindent As the assumption of periodicity is a simplifying approximation, we
introduce some erroneous effects because of it. These generally disappear as we
increase the number of particles (and hence the size of box), but the
computation time needed to run the simulations increase significantly with
increasing numbers. The purpose of the following analysis is to test the
\emph{relative} accuracy of different wave functions. With that in mind we have
used a small number of particles in the main results, where the absolute error
introduced is significant. The number should hopefully still be large enough to
introduce all the relevant effects and produce valid test results.


\subsection{Optimizing}

We have optimized~\cref{eq:McMillan-wave-function-def} using $\num{5000}$
iterations of $\num{1000}$ MC cycles each. We used standard Metropolis sampling
along with the ADAM optimizer. \cref{fig:He-benchmark-training} shows the
progression of both the energy and variational parameter during training.
Because of the strong correlations involved, there is significantly more
variance in these results compared to the benchmark used for Quantum Dots. We
see the value for $\beta$ oscillating without any indication of converging to a
fixed value.

\cref{tab:He-benchmark-results} show the energy estimates from the final model.
The same wave function (i.e. same value for $\beta$) has also been used on
different number of particles to illustrate how the energy increases for larger
systems. Based on computations with larger and larger systems, the value for $N
= 256$ is quite close to the apparent convergence for $N\to\infty$.

\begin{figure}[h]
  \centering
  % This file was created by matplotlib2tikz v0.7.4.
\begin{tikzpicture}

\definecolor{color0}{rgb}{0.12156862745098,0.466666666666667,0.705882352941177}

\begin{groupplot}[group style={group size=2 by 1}]
\nextgroupplot[
tick align=outside,
tick pos=left,
x grid style={white!69.01960784313725!black},
xlabel={\% of training},
xmin=-5, xmax=105,
xtick style={color=black},
ylabel={Ground state energy $[\si{\kelvin\per N}]$},
y grid style={white!69.01960784313725!black},
ymin=-6.85219168043208, ymax=-5.76372161210776,
ytick style={color=black},
ylabel near ticks,
]
\addplot [semithick, color0]
table {%
0 -5.81319752430432
1 -5.93452622568609
2 -6.00858962108258
3 -6.08665800506619
4 -6.20773427507232
5 -6.23156670870654
6 -6.25285373896968
7 -6.36683025632103
8 -6.36485217914965
9 -6.42748677702066
10 -6.50039764982493
11 -6.52719360265819
12 -6.56717088511529
13 -6.55506196487281
14 -6.60944841435202
15 -6.58072330140416
16 -6.60677852644892
17 -6.61647256171949
18 -6.70160055491517
19 -6.69863368699099
20 -6.69635189000038
21 -6.72593561696231
22 -6.68343964270742
23 -6.7429087393482
24 -6.71851493402437
25 -6.7345213413159
26 -6.71034228662546
27 -6.74732848543959
28 -6.72175233470759
29 -6.75688969206912
30 -6.74660417455725
31 -6.72846362943138
32 -6.7587149797462
33 -6.75693148879914
34 -6.76258422232693
35 -6.75085981956242
36 -6.74387871373416
37 -6.74109156636538
38 -6.77243478754695
39 -6.74052798320351
40 -6.74424094866635
41 -6.76775739672773
42 -6.7559919917146
43 -6.75039426906645
44 -6.75370511228573
45 -6.75896538263834
46 -6.74032690549759
47 -6.78997769653142
48 -6.73115803993219
49 -6.75099235927379
50 -6.76689812871345
51 -6.74849882697142
52 -6.7674145141362
53 -6.78131648574984
54 -6.75906095822701
55 -6.77784471235837
56 -6.79117471226126
57 -6.80271576823552
58 -6.77509213456717
59 -6.78716474735426
60 -6.74039641661558
61 -6.79239158305779
62 -6.75498454590202
63 -6.7634873982812
64 -6.73744637804431
65 -6.75774478925854
66 -6.77807578595877
67 -6.76413401844244
68 -6.73777049704596
69 -6.76212594901964
70 -6.75869381525464
71 -6.75888649214504
72 -6.74197262277744
73 -6.76612166760104
74 -6.75684274744045
75 -6.7378847148746
76 -6.74540734816332
77 -6.71361620588114
78 -6.75982157137573
79 -6.75084877076814
80 -6.762793676612
81 -6.74920050658278
82 -6.74846814015804
83 -6.77019406086908
84 -6.7484431760123
85 -6.76377512756983
86 -6.77395128912729
87 -6.77676908332129
88 -6.73673374230908
89 -6.76674164837398
90 -6.78038626616403
91 -6.77584777447017
92 -6.71076974724152
93 -6.76035815397443
94 -6.76350918211843
95 -6.76389862072772
96 -6.77110719678763
97 -6.75714843483792
98 -6.77043867917657
99 -6.71547668584384
100 -6.75350211816952
};

\nextgroupplot[
legend cell align={left},
legend style={at={(0.03,0.97)}, anchor=north west, draw=white!80.0!black},
tick align=outside,
tick pos=left,
x grid style={white!69.01960784313725!black},
xlabel={\% of training},
xmin=-5, xmax=105,
xtick style={color=black},
y grid style={white!69.01960784313725!black},
ymin=2.84311677655944, ymax=2.99454769225178,
ytick style={color=black},
ylabel={Variational parameter $\beta$ $[\si{\angstrom}]$},
ylabel style={rotate=-180},
ylabel near ticks,
ytick pos=right,
ylabel style={rotate=-180},
]
\addplot [semithick, color0]
table {%
0 2.85
1 2.85771877343818
2 2.8650426908874
3 2.87229961410334
4 2.87926515215211
5 2.88600281382441
6 2.89220140344377
7 2.89851875286748
8 2.90461755434636
9 2.91032476320017
10 2.91576989197661
11 2.92094435853081
12 2.9257935546088
13 2.93056536447931
14 2.93519788266342
15 2.93954634850754
16 2.94358583039877
17 2.94736833558584
18 2.95086530798089
19 2.95416009410758
20 2.95708246669139
21 2.96029424262311
22 2.96278729802401
23 2.96542975622013
24 2.96776088038637
25 2.96983502291505
26 2.97132900145053
27 2.97319498626816
28 2.97425428304438
29 2.97573496432073
30 2.97692893065558
31 2.97799918989409
32 2.9790535423368
33 2.98027251536911
34 2.98139610222615
35 2.98216419975245
36 2.98245071983135
37 2.98261528393402
38 2.98317005753593
39 2.98285828689244
40 2.98328409623724
41 2.98327184277541
42 2.98420872487105
43 2.98431102032459
44 2.98405545942704
45 2.98402716102763
46 2.98451107624493
47 2.98442284296482
48 2.98439045174919
49 2.98511270289013
50 2.98579429264609
51 2.98567717994044
52 2.98552782748783
53 2.98537803518609
54 2.98563125652273
55 2.98528049661546
56 2.98571231758376
57 2.98452772127367
58 2.98503960591317
59 2.98510299685132
60 2.98422617030646
61 2.98458841600336
62 2.98438973868782
63 2.98391334292601
64 2.98392097606065
65 2.98348805264952
66 2.98369115789756
67 2.98414733320096
68 2.98384692930286
69 2.98464229582106
70 2.98463036262134
71 2.98373567379686
72 2.98371158378309
73 2.98488136139775
74 2.9846521395285
75 2.9844328413153
76 2.98505392018769
77 2.98615577858719
78 2.98685090647166
79 2.98766446881122
80 2.98694631552791
81 2.98518517959941
82 2.98491799512055
83 2.98543572667322
84 2.98534325008699
85 2.98519634670357
86 2.98435857464363
87 2.98409743281243
88 2.9841897962539
89 2.98410842090462
90 2.98508744369597
91 2.98502255078322
92 2.98467748255916
93 2.98470976363276
94 2.98553663644782
95 2.98647577600856
96 2.98416967633852
97 2.98447136823118
98 2.98537649056117
99 2.98571613177779
100 2.9855315339833
};
\end{groupplot}

\end{tikzpicture}
  \caption{Helium benchmark training.}
  \label{fig:He-benchmark-training}
\end{figure}

\begin{table}[h]
  \centering
  \begin{tabular}{lS[table-format=1.5]*2{S[table-format=1.3]}*2{S[table-format=1.1]}}
\toprule
\addlinespace
& {$\langle E_L\rangle$} & {CI$^{95}_-$} & {CI$^{95}_+$} & {Std} & {Var} \\
\addlinespace
\midrule
\addlinespace
\addlinespace
    $\psi_M^{(32)}$ & -6.76(1) & -6.79 & -6.74 & \num{4.6e-01} & \num{2.1e-01}\\
$\psi_M^{(64)}$ & -6.15(1) & -6.18 & -6.12 & \num{3.0e-01} & \num{9.1e-02}\\
$\psi_M^{(256)}$ & -5.848(10) & -5.867 & -5.829 & \num{2.2e-01} & \num{4.7e-02}\\
\addlinespace\addlinespace\bottomrule
\end{tabular}
  \caption{Helium benchmark results}
  \label{tab:He-benchmark-results}
\end{table}


\begin{itemize}
\item Show plot of training for 32 particles
\item In table of energies, include same wave function on larger numbers (64,
  128, 256)
\end{itemize}


\end{document}
