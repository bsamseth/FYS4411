\documentclass[Thesis.tex]{subfiles}
\begin{document}
\chapter{Appendix}

\section{Derivatives for the Symmetric RBM}

The standard RBM definition is stated for reference:
\begin{align}\label{eq:rbm-re-def}
    \Psi(\mathbf{X}) &= \exp(-\sum_{i=1}^M \frac{(X_i - a_i)^2}{2\sigma^2} )
                        \prod_{j=1}^N\qty[1 + \exp(b_j + \sum_{i=1}^M \frac{X_i
                        W_{ij}}{\sigma^2})]\\
    &\defeq \exp(-\sum_{i=1}^Mu_i)\prod_{j=1}^N\qty(1 + \exp(v_j)).
\end{align}

We now define that the RBM has $M$ degrees of freedom, as before, and that each
particle has $f$ of them related to its coordinates. In this case, for the RBM
to be symmetric upon the exchange of particles, each group of $f$ entries in
$\vb X$ should be treated equally within the wavefunction.

We define $\vb b\in \mathds{R}^{N}$ as before, i.e. one bias value per hidden
node in the RBM. Let $\vb a\in \mathds{R}^{f}$ be a vector of biases, with one
value \emph{per coordinate direction}. Further, let $\vb W\in \mathds{R}^{f
\times N}$ be a matrix of weights. The weights connect each coordinate to every
hidden node. For notational breviety, we also denote $c(i) \defeq i \pmod{f} + 1$. We
define the symmetric version of the RBM then as follows:

\begin{align}\label{eq:rbm-sym-def}
    \Psi_S(\mathbf{X}) &=\exp(-\sum_{i=1}^M \frac{\qty(X_i - a_{c(i)})^2}{2\sigma^2} )
                        \prod_{j=1}^N\qty[1 + \exp(b_j + \sum_{i=1}^M \frac{X_i
                        W_{c(i)j}}{\sigma^2})]\\
    &\defeq \exp(-\sum_{i=1}^Mu_i')\prod_{j=1}^N\qty(1 + \exp(v_j')).
\end{align}
We notice that \autoref{eq:rbm-sym-def} is equivalent to \autoref{eq:rbm-re-def}
when $f=M$, and that all that has changed is the subsittution $i\rightarrow
c(i)$. The symmetry of this function is now clear, seing as shifting any index
$i\rightarrow i + k\times f$ for some integer factor $k$ will result in the same
output. This operation is equivalent to changing the order of the particles.

We now derive the explicit derivative that are needed in within the VMC
framework.

\begin{align}
    \frac{1}{\Psi_S} \pdv{\Psi_S}{a_{c(k)}} &= \pdv{\ln
    \Psi_S}{\Psi_S}\pdv{\Psi_S}{a_{c(k)}} = \pdv{\ln \Psi}{a_{c(k)}}\\
    &= \pdv{}{a_{c(k)}}\qty[ -\sum_{i=1}^M \frac{\qty(X_i -
    a_{c(i)})^2}{2\sigma^2} + \text{Constant w.r.t. } a_{c(i)} ]\\
    &= \sum_{i=k\atop{c(i)=c(k)}}^M \frac{X_i -
    a_{c(i)}}{\sigma^2}.\label{eq:rbm-sym-deriv-a}
\end{align}
where the last sum indicates a sum over all $i$ where $1 \leq i \leq M$ such
that $c(i) = c(k)$. All these terms will contain the same bias $a_{c(k)}$.

Now the hidden node biases:

\begin{align}
    \frac{1}{\Psi_S} \pdv{\Psi_S}{b_k} &= \pdv{\ln\Psi_S}{b_j}\\
    &= \pdv{}{b_k}\qty[\text{Constant w.r.t. }b_k +
    \sum_{j=1}^N\ln\qty(1+\exp(v_j'))]\\
    &= \frac{\exp(v_k')}{1 + \exp(v_k')} = \frac{1}{1 + \exp(-v_k')}\label{eq:rbm-sym-deriv-b}
\end{align}
which is the same form as for the regular RBM, as we would expect because the
hidden node biases are defined the same way.

Lastly we consider the weights:
\begin{align}
    \frac{1}{\Psi_S} \pdv{\Psi_S}{W_{kl}} &= \pdv{\ln\Psi_S}{W_{kl}}\\
    &= \pdv{}{W_{kl}}\qty[\text{Constant w.r.t. }W_{kl} +
    \sum_{j=1}^N\ln\qty(1+\exp(v_j'))]\\
    &= \frac{\exp(v_l')}{\ln(1 + \exp(v_l'))} \pdv{v_l'}{W_{kl}}\\
    &= \frac{1}{1 + \exp(-v_l')}\sum_{i=k\atop{c(i)=c(k)}}^M
    \frac{X_i}{\sigma^2} \label{eq:rbm-sym-deriv-W}
\end{align}
which is simply the sum over all suitable $k$'s of the derivatives we obtain
from the regular RBM.

Finally, we shall compute the first- and second order derivatives needed. Before
jumping into this, we can note that the transformation $\Psi\right\Psi_S$ really
only changed the indices of the weights and biases, but did not change how the
$X_i$ components appear in the wavefunction. If we were to start carying out the
same precedure as done for RBM in ??, we would quickly see that we end with the
same result, s.t. the same transformation. We obtain

\begin{align}
    \frac{1}{\Psi_S} \pdv{\Psi_S}{X_k} &= \frac{1}{\sigma^2}\qty(a_{c(k)}-x_k +
    \sum_j^N \frac{w_{c(k)j}}{1+e^{-v_j'}}).\label{eq:rbm-sym-first-order-deriv}
\end{align}
\begin{align}
    \begin{split}
    \frac{1}{\Psi_S}\pdv[2]{\Psi_S}{X_k}
        &= -\frac{1}{\sigma^2} + \sum_j^N
        \frac{w_{c(k)j}^2}{\sigma^4}
        \frac{e^{-v_j'}}{\qty(1+e^{-v_j'})^2} \\
        &\quad{   } +
        \frac{1}{\sigma^4}\qty(a_{c(k)}-x_k+\sum_j^N
        \frac{w_{{c(k)}j}}{1+e^{-v_j'}})^2
    \end{split}\label{eq:rbm-sym-second-deriv}
\end{align}

\end{document}
