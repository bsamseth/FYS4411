\documentclass[Thesis.tex]{subfiles}
\begin{document}
\chapter{Introduction}
\label{chp:introduction}

\Gls{qm} is the fundamental theory which describes nature at its
smallest scales. Its predictions have been verified by every experiment ever
devised to test it, and is considered by many to be our most successful physical
theory. Still, \gls{qm} is far from fully understood, and there is an ocean of
questions we still want to get out of it. Continued study of \gls{qm} is arguably one
of the most worthwhile undertakings we can work on.

Nevertheless, to a layperson \acrshort{qm} might just mean \say{complicated science} and it
is happily ignored in favor of more relatable topics. But if we could turn off \gls{qm}
for a day, its effects on our daily life would become blazingly obvious. You no
longer need to go to that doctors appointment for an \acrshort{mri}, because that not a
thing anymore. Better call and cancel - to bad your phone is now the size of
your apartment. At least you can still watch the world fall apart on your new TV -
never mind, \acrshort{led}s don't work anymore either.

The point of this thought experiment is not to say everyone should become a
theoretical physicist, but rather to point out the immense technological
progress we have made as side effects of our pursuit to unravel natures inner
workings. So while \acrshort{qm} may be the law of the microscopic, there are
real life, large scale effects from which we benefit on a daily basis.\\

The most challenging problem in \gls{qm} is the study of strongly correlated
many-body systems. Put 100 electrons in a box and press play. What happens next
turns out to be extremely complicated to accurately predict. The equation that
governs this world is the Schrödinger equation,

\begin{align}
  \label{eq:schrodinger-intro-def}
  \hat H\ket{\Psi} &= i\hbar \pdv{}{t}\ket\Psi.
\end{align}
We will revisit this equation in \cref{chp:the-quantum-problem}, but for now it
suffices to know that $\ket\Psi$ is the so called \emph{wave function}. This
mysterious thing is what we seek to get out of \cref{eq:schrodinger-intro-def},
and it can tell us \emph{anything} we wish to know about the system.

The reason for our troubles is that \cref{eq:schrodinger-intro-def} leads us to
a many-body wave function plagued with \emph{exponential} scaling complexity.
For instance, the wave function might require $2^N$ components, which for our
100 electrons means we would need a \emph{nonillion} ($10^{30}$) parts. That's
more parts than there are bits in all the computers of the world, with plenty to
spare.

\begin{comment}
In most other areas of science we have been able to tame the mathematical
challenges through sophisticated high-performance computational frameworks.
Scaling complexities of $N^2$, $N^3$ or similar can usually be overcome by
simply using bigger and better computers, all the way to supercomputers with
hundreds of thousands of cores. But exponential scaling - that becomes
infeasible quickly.
\end{comment}

In the words of one of the founding fathers of \acrshort{qm}:
\begin{displayquote}
\emph{The underlying physical laws necessary for the mathematical theory of a large
part of physics and the whole of chemistry are thus completely known, and the
difficulty is only that the exact application of these laws leads to equations
much too complicated to be soluble.} --- P.A.M. Dirac
\end{displayquote}


So how do we do get around this fundamental issue of an equation so complicated
we literally cannot solve it through analytical means? Throughout this thesis,
we will be concerned with one particular method: \emph{Guessing}. Based on our
theoretical understanding we can make an educated guess as to what $\ket\Psi$
should look like. Even if the guess is not completely accurate, we can now
simulate the system as if it was described by this wave function and measure its
properties. From fundamental principles, we know nature prefers the state of
smallest possible energy. That means that if we tweak our guess slightly and
obtain a smaller predicted energy, then our new guess is closer to the
underlying truth. If we keep updating our guess, keeping those that lead to
smaller energies, we will eventually end up with something that hopefully is a
much better approximate solution. The problem is now how to make good guesses,
and how tweak them in the right direction. This approach is known as \gls{vmc},
and will be discussed in depth in \cref{chp:variational-monte-carlo}.\\

Enter the world of \gls{ml} which over the last years have seen
huge successes in a range of real-world problems, including computer vision and
natural language processing, each with a scaling complexity similar to
the quantum many-body problem. Physicists have taken notice and are now trying
to apply techniques from \gls{ml} in an attempt to make progress in \gls{qm}. \Acrlong{ml} has
demonstrated how we can train computer models to see patterns and connections
far beyond human comprehension, and be able to condense these down to tangible
predictions that we can use.

The goal of this thesis is to build upon existing machinery for \gls{vmc} by extending
theoretical guesses for $\ket\Psi$ with a neural network component, thus massively
increasing the flexibility and expressiveness of our guesses. If successful, this
network will be able to learn the physical correlations present in the system
and correct for them more accurately than we could ever do ourselves.



\section*{How to Use This Thesis}

This thesis is written with an undergraduate physics student as the intended
audience. It is written in a way that I would have liked to read it when first
embarking on a master's degree. Regardless of how many readers fall into this
category, it is our hope that anyone comfortable with first year university
level mathematics should able to understand the majority of the results of this
thesis, even if not every detail is equally clear. While the topics at hand might be
inherently complex, we have made an attempt at guiding the reader through the
background in a way that explains all necessary components as they arise.


\subsection*{Structure}
\cref{prt:theory} starts with a introduction to all the underlying theory
necessary to understand the results of this work. \cref{chp:the-quantum-problem}
starts with introducing the relevant bits of \acrlong{qm}, followed
immediately by a presentation of \acrlong{vmc} and \acrlong{mc} methods
in general (\cref{chp:variational-monte-carlo,chp:monte-carlo}). This forms the
backbone of the thesis. Next we shift gears and discuss \acrlong{ml} in
general (\cref{chp:machine-learning}), focusing again on the parts which are
most relevant to us. We end \cref{prt:theory} by talking about the glue that
connects all of this together (\cref{chp:mergin-vmc-with-ml}).

\cref{prt:implementation} is all about the technical details of how we implement
the algorithms from \cref{prt:theory} into efficient and correct code. We
discuss design choices (\cref{chp:parallelization,chp:auto-diff}), and culminate
with presenting QFLOW, the library we have
developed for all our computing needs (\cref{chp:qflow}). Lastly we make an effort to convince the
reader that the code is correct by verifying the results on some
selected idealized cases (\cref{chp:verfication}).

\cref{prt:results} finally presents the results of the new method we have
developed, testing it on both a few-body system (\cref{chp:quantum-dots}) and a
more complicated many-body system (\cref{chp:liquid-helium}).

Finally, \cref{prt:conclusion} offers conclusions and future prospects.

\subsection*{Reproducible}

One of the most frustrating parts of writing this thesis has been attempting to
reproduce results from published articles. Far to often vital details are left
out, both in theory and implementation, leaving poor soles to guess as to how
they achieved the results they present.

We have made a conscious effort to be
better in this regard. To that end, all of the code, data, figures etc.\ that
you find in this thesis are openly available at one central
repository~\cite{qflow}. That even includes the source code for this very
document. The QFLOW library is permissively licensed under the MIT license,
allowing anyone to use the code how ever they see fit. Furthermore, \emph{every} table
and figure has a reference to where you may find the exact source code which
generated it, leaving no doubt as to the details of any results.

\subsection*{Notation}

Another point of possible frustration is unclear use of notation. While we
assume a certain general familiarity with mathematics, we strive to make our
notation as clear as possible. All symbols should an accompanying
explanation following their first use. Furthermore, while we attempt to use
standard notation where possible, any doubt should be removed by consulting the
notation reference in \cref{app:notation-reference}.

\end{document}
